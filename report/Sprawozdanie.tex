\documentclass[a4paper,titlepage,11pt,floatssmall]{mwrep}
\usepackage[left=2.5cm,right=2.5cm,top=2.5cm,bottom=2.5cm]{geometry}
\usepackage[OT1]{fontenc}
\usepackage{polski}
\usepackage[utf8]{inputenc}
\usepackage{amsmath}
\usepackage{amssymb}
\usepackage{graphicx}
\usepackage{rotating}
\usepackage{pgfplots}
\usetikzlibrary{pgfplots.groupplots}

\usepackage{siunitx}

\usepackage{float}
\definecolor{szary}{rgb}{0.95,0.95,0.95}
\sisetup{detect-weight,exponent-product=\cdot,output-decimal-marker={,},per-mode=symbol,binary-units=true,range-phrase={-},range-units=single}

\SendSettingsToPgf
\title{\bf Sprawozdanie z projektu nr 2\\ Zadanie nr 36  \vskip 0.1cm}
\author{Jakub Sikora \and nr albumu 283418}
\date{\today}
\pgfplotsset{compat=1.15}	
\begin{document}


\makeatletter
\renewcommand{\maketitle}{\begin{titlepage}
		\begin{center}{\LARGE {\bf
					Wydział Elektroniki i Technik Informacyjnych}}\\
			\vspace{0.4cm}
			{\LARGE {\bf Politechnika Warszawska}}\\
			\vspace{0.3cm}
		\end{center}
		\vspace{5cm}
		\begin{center}
			{\bf \LARGE Modelowanie i identyfikacja \vskip 0.1cm}
		\end{center}
		\vspace{1cm}
		\begin{center}
			{\bf \LARGE \@title}
		\end{center}
		\vspace{2cm}
		\begin{center}
			{\bf \Large \@author \par}
		\end{center}
		\vspace*{\stretch{6}}
		\begin{center}
			\bf{\large{Warszawa, \@date\vskip 0.1cm}}
		\end{center}
	\end{titlepage}
	}
\makeatother
\maketitle

\tableofcontents

\chapter{Zadania obowiązkowe}

\section{Identyfikacja modeli statycznych}

\subsection{Wykres danych statycznych}

\indent Na rysunku poniżej przedstawiłem dane z pliku \texttt{danestat36.txt}. Na osi poziomej przedstawiłem sterowanie które zmienia się w przedziale $[-1; 1]$. Na osi pionowej wykresu znajduje się wyjście procesu które zmienia się w przedziale $[-3,1; 0,25]$.
\bigskip
\begin{figure}[H]
\centering
\includegraphics[width = \textwidth]{../figures/stat/dane_overall.pdf}
\caption{Przedstawienie statycznych danych pomiarowych}
\end{figure}


\newpage
\subsection{Podział na zbiór uczący i weryfikujący}
\indent W celu identyfikacji modelu metodą najmniejszych kwadratów, najpierw należy dokonać podziału zbioru danych na zbiór uczący i zbiór weryfikujący. Najprostszą metodą podziału, z której powinniśmy uzyskać dwa równomierne zbioru, jest przyporządkowywanie rekordu na zmianę raz do zbioru uczącego a raz do zbioru weryfikującego. Za pomocą matlaba (skrypt \texttt{stat/zad\_{}b.m} dokonałem prostego podziału. Oba zbiory przedstawiłem na rysunkach poniżej.

\begin{figure}[H]
\centering
\includegraphics[width = 0.75\textwidth]{../figures/stat/dane_ucz.pdf}
\caption{Zbiór danych uczących}
\end{figure}

\begin{figure}[H]
\centering
\includegraphics[width = 0.75\textwidth]{../figures/stat/dane_wer.pdf}
\caption{Zbiór danych weryfikujących}
\end{figure}

\begin{figure}[H]
\centering
\includegraphics[width = \textwidth]{../figures/stat/dane_comp.pdf}
\caption{Porównanie zbiorów: weryfikującego i uczącego.}
\end{figure}

\indent Na podstawie rysunku 1.4, można stwierdzić że uzyskany podział jest wystarczająco równomierny. 







\end{document}
